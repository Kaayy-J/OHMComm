\documentclass[11pt,a4paper]{report} 
\usepackage[ngerman]{babel} % deutsch und deutsche Rechtschreibung
\usepackage[utf8]{inputenc} % Unicode Text 
\usepackage[T1]{fontenc} % Umlaute und deutsches Trennen
\usepackage{textcomp} % Sonderzeichen wie Euro, Copyright, etc.
\usepackage[hyphens]{url} % Hyperlinks, eMail-Adressen, Pfadangaben
\usepackage{amssymb} % mathmatische Symbole
\usepackage{microtype} % reguliert Abstände zwischen Buchenstaben
\usepackage{graphicx} % wir wollen Bilder einfügen
\usepackage{float} % Umgebung die sich automatisch im Dokument an passenden Positionen bewegen (floaten)
\usepackage{emptypage} % Wirklich leer bei leeren Seiten

\usepackage{listings} % schöne Quellcode-Listings
% ein paar Einstellungen für akzeptable Listings
\lstset{basicstyle=\ttfamily, columns=[l]flexible, mathescape=true, showstringspaces=false, numbers=left, numberstyle=\tiny}
\lstset{language=C++} % Set your language (you can change the language for each code-block optionally)
% more information about listings: https://en.wikibooks.org/wiki/LaTeX/Source_Code_Listings

% Spezialpakete
\usepackage{epigraph} % Zum Positionieren von Bemerkungen links, rechts, oben und unten vom Text
\setlength{\epigraphrule}{0pt} % kein Trennstrich

% Seitenlayout
\usepackage[paper=a4paper,width=14cm,left=35mm,height=22cm]{geometry}
\usepackage{setspace}
\linespread{1.15}
\setlength{\parskip}{0.5em}
\setlength{\parindent}{0em} % im Deutschen Einrückung nicht üblich, leider

\newcommand{\phv}{\fontfamily{phv}\fontseries{m}\fontsize{9}{11}\selectfont}
\usepackage{fancyhdr} % ermöglicht schickere Header und Footer
\pagestyle{fancy}
\renewcommand{\chaptermark}[1]{\markboth{#1}{}}
\fancyhead[L]{\phv \leftmark}
\fancyhead[RE,LO]{\phv \nouppercase{\leftmark}}
\fancyhead[LE,RO]{\phv \thepage}
% Unten besser auf alles Verzichten
%\fancyfoot[L]{\textsf{\small \kurztitel}}
\fancyfoot[C]{\ } % keine Seitenzahl unten
%\fancyfoot[R]{\textsf{\small Medieninformatik}}

% Quellen aufteilen z.B. in Online-Quellen und Literaturverzeichnis
\usepackage{bibtopic} 

% Config 1: Times New Roman, gewohnter Font, ok tt und serifenlos
%\usepackage{mathptmx} 
%\usepackage[scaled=.95]{helvet}
%\usepackage{courier}

% Config 2: Palatino mit guten Fonts für tt und serifenlos
\usepackage{mathpazo} % Palatino, mal was anderes
\usepackage[scaled=.95]{helvet}
\usepackage{palatino }

% Config 3: New Century Schoolbook sieht auch nett aus (macht auch tt und serifenlos)
%\usepackage{newcent}

% Mehr Informationen zu Fonts: https://de.sharelatex.com/learn/Font_typefaces


% Zum Zeigen von Fehlern
\usepackage{soulutf8}
\newcommand*\falsch{\st}

% damit wir nicht so viel tippen müssen, nur für Demo 
\usepackage{blindtext} 

% Float-Objekte sollen die Section nicht verlassen in der sie eingefügt worden sind
\usepackage[section]{placeins}

\begin{document}

\begin{titlepage}
  \begin{center}
    % Kopf der Seite
    \parbox[t]{8cm}{
      % \textsf würde das Aussehen der ersten Seite ruinieren, 
      % wer will, soll das selbst außen rum machen...
      TH Nürnberg Georg Simon Ohm\\
      Fakultät Informatik \\
	}
    \vfill    
    {\LARGE Projektbericht} \\[0.5cm]
    {\large im Rahmen des Moduls IT-Projekt} \\[5mm]
    \rule{\textwidth}{1pt}\\[0.5cm]
    {\begin{spacing}{1.15} \huge \bfseries OHMComm \\Plattformunabhängiges Framework zur Audioübertragung \\ \end{spacing}}
    \rule{\textwidth}{1pt}    
    \vfill    
    \begin{tabular}{ll} % Mitte der Seite
      Vorgelegt von & Daniel, Jonas, Kamal \\
      am & bald \\
      Betreuer & Prof. Dr. M. Tessmann \\
    \end{tabular}    
    \vfill
\end{center}
\end{titlepage}
\cleardoublepage

% Zusammenfassung
\begin{abstract} 
OHMComm ist ein Framework zur Audiokommunikation und -verarbeitung, dass im Rahmen des IT-Projekts an der Technischen Hochschule Nürnberg entwickelt wurde.
\end{abstract}

\tableofcontents




\chapter{Einleitung}
\section{Ziel}
\section{Aufbau des Berichts}

\chapter{Anforderungsanalyse}
\section{Funktionale und Nicht-Funktionale Anforderungen}
\section{Datenhaushalt}
\section{Alternative Softwarelösungen}

\chapter{Entwurf}
\section{Projektverwaltung und Werkzeuge}
\section{Build Prozess}
\section{Externe Softwarekomponenten}

\section{Softwarearchitektur}
\subsubsection{RTAudio}
\subsubsection{Opus}
\subsubsection{CPPTest}
\subsection{Verarbeitungskette}
\subsection{Instanziierung}
\subsection{Statistiken}
\subsection{Einstellungsmöglichkeiten und Konfigurationen}
\subsection{RTP-Protokoll}
\subsubsection{RTCP-Protokoll}
\subsection{RTP-Buffer}

\chapter{Implementierung}

\chapter{Prototypische Anwendungen}
\section{Voice-over-IP Konsolenanwendungen}
\subsection{Ziel der Anwendung}
\subsection{Steuerung}
\subsection{Anwendungen ordnungsgemäß beenden}
\section{Filerecorder}

\chapter{Fazit und Ausblick}



















\bibliographystyle{apalike} % Literaturverzeichnis
\begin{btSect}{thesis} % mit bibtopic Quellen trennen
\section*{Literaturverzeichnis}
\btPrintCited
\end{btSect}
\begin{btSect}{online}
\section*{Online-Quellen}
\btPrintCited
\end{btSect}

\end{document}
