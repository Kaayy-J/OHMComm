\documentclass[11pt,a4paper]{report} 
\usepackage[ngerman]{babel} % deutsch und deutsche Rechtschreibung
\usepackage[utf8]{inputenc} % Unicode Text 
\usepackage[T1]{fontenc} % Umlaute und deutsches Trennen
\usepackage{textcomp} % Sonderzeichen wie Euro, Copyright, etc.
\usepackage[hyphens]{url} % Hyperlinks, eMail-Adressen, Pfadangaben
\usepackage{amssymb} % mathmatische Symbole
\usepackage{microtype} % reguliert Abstände zwischen Buchenstaben
\usepackage{graphicx} % wir wollen Bilder einfügen
\usepackage{float} % Umgebung die sich automatisch im Dokument an passenden Positionen bewegen (floaten)
\usepackage{emptypage} % Wirklich leer bei leeren Seiten

\usepackage{hyperref} % macht ToC und alle internen Lonks klickbar (für leichtere Navigation)
\hypersetup{
    colorlinks,
    citecolor=black,
    filecolor=black,
    linkcolor=black,
    urlcolor=black
}

\usepackage{listings} % schöne Quellcode-Listings
% ein paar Einstellungen für akzeptable Listings
\lstset{basicstyle=\ttfamily, columns=[l]flexible, mathescape=true, showstringspaces=false, numbers=left, numberstyle=\tiny}
\lstset{language=C++} % Set your language (you can change the language for each code-block optionally)
% more information about listings: https://en.wikibooks.org/wiki/LaTeX/Source_Code_Listings

% Spezialpakete
\usepackage{epigraph} % Zum Positionieren von Bemerkungen links, rechts, oben und unten vom Text
\setlength{\epigraphrule}{0pt} % kein Trennstrich

% Seitenlayout
\usepackage[paper=a4paper,width=14cm,left=35mm,height=22cm]{geometry}
\usepackage{setspace}
\linespread{1.15}
\setlength{\parskip}{0.5em}
\setlength{\parindent}{0em} % im Deutschen Einrückung nicht üblich, leider

\newcommand{\phv}{\fontfamily{phv}\fontseries{m}\fontsize{9}{11}\selectfont}
\usepackage{fancyhdr} % ermöglicht schickere Header und Footer
\pagestyle{fancy}
\renewcommand{\chaptermark}[1]{\markboth{#1}{}}
\fancyhead[L]{\phv \leftmark}
\fancyhead[RE,LO]{\phv \nouppercase{\leftmark}}
\fancyhead[LE,RO]{\phv \thepage}
% Unten besser auf alles Verzichten
%\fancyfoot[L]{\textsf{\small \kurztitel}}
\fancyfoot[C]{\ } % keine Seitenzahl unten
%\fancyfoot[R]{\textsf{\small Medieninformatik}}

% Quellen aufteilen z.B. in Online-Quellen und Literaturverzeichnis
\usepackage{bibtopic} 

% Config 1: Times New Roman, gewohnter Font, ok tt und serifenlos
%\usepackage{mathptmx} 
%\usepackage[scaled=.95]{helvet}
%\usepackage{courier}

% Config 2: Palatino mit guten Fonts für tt und serifenlos
\usepackage{mathpazo} % Palatino, mal was anderes
\usepackage[scaled=.95]{helvet}
\usepackage{palatino }

% Config 3: New Century Schoolbook sieht auch nett aus (macht auch tt und serifenlos)
%\usepackage{newcent}

% Mehr Informationen zu Fonts: https://de.sharelatex.com/learn/Font_typefaces


% Zum Zeigen von Fehlern
\usepackage{soulutf8}
\newcommand*\falsch{\st}

% damit wir nicht so viel tippen müssen, nur für Demo 
\usepackage{blindtext} 

% Float-Objekte sollen die Section nicht verlassen in der sie eingefügt worden sind
\usepackage[section]{placeins}

\begin{document}

\begin{titlepage}
  \begin{center}
    % Kopf der Seite
    \parbox[t]{8cm}{
      % \textsf würde das Aussehen der ersten Seite ruinieren, 
      % wer will, soll das selbst außen rum machen...
      TH Nürnberg Georg Simon Ohm\\
      Fakultät Informatik \\
	}
    \vfill    
    {\LARGE Projektbericht} \\[0.5cm]
    {\large im Rahmen des Moduls IT-Projekt} \\[5mm]
    \rule{\textwidth}{1pt}\\[0.5cm]
    {\begin{spacing}{1.15} \huge \bfseries OHMComm \\Plattformunabhängiges Framework zur Audioübertragung \\ \end{spacing}}
    \rule{\textwidth}{1pt}    
    \vfill    
    \begin{tabular}{ll} % Mitte der Seite
      Vorgelegt von & Daniel Stadelmann, Jonas Ziegler, Kamal Bhatti \\
      am & \today \\
      Betreuer & Prof. Dr. M. Tessmann \\
    \end{tabular}    
    \vfill
\end{center}
\end{titlepage}
\cleardoublepage

% Zusammenfassung
\begin{abstract} 
OHMComm ist ein Framework zur Audiokommunikation und -verarbeitung, dass im Rahmen des IT-Projekts an der Technischen Hochschule Nürnberg entwickelt wurde.
\end{abstract}

\tableofcontents

%Teil Inhalt auf, für bessere Modularität und paralleles Schreiben
\chapter{Einleitung}
\section{Was ist OHMComm?}
OHMComm ist ein IT-Projekt der Informatik Fakultät an der technischen Hochschule in Nürnberg. Das Projekt wird im Rahmen des Bachelorstudienganges Informatik umgesetzt und erstreckt sich über zwei Semester. Herr Prof. Dr. M. Tessmann ist Projektinitiator und Projektleiter. Das Ziel des Projektes ist die Erstellung eines plattformunabhängiges Audiokommunikationsframework. Das Framework soll sämtliche Funktionalität, die für eine erfolgreiche Kommunikation zwischen zwei Teilnehmern benötigt wird, zur Verfügung stellen. Das Framework als auch das Projekt tragen den Namen OHMComm. Der Anwendungsfall für die Hochschule ist der Einsatz der Software in Lehrveranstaltungen, jedoch sind auch andere Anwendungsszenarien denkbar, wie z.B. die Integration in externe Softwareanwendungen.

\section{Projektbeschreibung}
Die Projektbeschreibung von Herr Prof. Dr. M. Tessmann beschreibt die Ziele, die im Rahmen des IT-Projektes erreicht werden sollten. Der obligatorische Funktionsumfang des Audiokommunikationsframework lässt sich in vier größere Bereiche zerlegen. Zum einem wird für die Audioaufnahme und Wiedergabe eine Schnittstelle zur Audiohardware benötigt. Der zweite Block beinhaltet die Kodierung und Dekodierung von Audiodaten. Diese Funktionalität darf man keineswegs als optional betrachten, da in der Praxis eine unkomprimierte Audioübertragung nicht machbar ist. Die meisten Leitungen verfügen hierfür nicht über die notwendige Bandbreite. Das Versenden und Empfangen von Audiodaten mit dem User Datagram Protocol (UDP) und Real-Time-Transport-Protokol (RTP) ist ein weiterer großer Implementierungsblock. RTP stellt ein standardisiertes Protokoll für die Audio- und Videoübertragung dar, welches 1996 von der Audio-Video Transport Working Group of the Internet Engineering Task Force (IETF) entwickelt wurde. Der letzte Block umfasst das Erstellen eines Jitterbuffers. Durch das Versenden von UDP-Paketen kann die Paketreihenfolge beim Empfänger nicht gewährleistet. Die Aufgabe des Buffers ist es, durch eine minimale Verzögerung vor dem Abspielen der Audiodaten, die Paketreihenfolge wiederherzustellen. Dadurch lässt sich die Audioqualität wesentlich erhöhen. Sämtliche Funktionen des Frameworks sollen in einer Beispielanwendung getestet werden, welche ebenfalls ein Teil des Projektes ist. Ferner soll es möglich statistische Werte des Frameworks zu sammeln, um es entsprechend bewerten zu können. Latenzen spielen in der Audioverarbeitung eine wesentliche Rolle, deshalb hat der Punkt Performanz eine besonderen Stellenwert. Auch das Ziel der Plattformunabhängigkeit ist ein wesentliches Ziel des Frameworks, das umgesetzt werden soll. Außerdem soll das Framework den Prinzipien des objektorientierten Designs entsprechen.
	
\section{Aufbau des Berichts}
Der Aufbau des Berichts entspricht den klassischen Phasen der Softwareentwicklung und ist unterteilt in den Kapiteln Anforderungsanalyse, Entwurf, Implementierung, Prototypische Voice-over-IP Konsolenanwendung und dem Fazit und Ausblick. 
In der Anforderungsanalyse werden aus den Projektzielen die konkreten Anforderungen ermittelt und diese in funktionale und nicht-funktionale Anforderungen klassifiziert. Außerdem wird hier definiert, wie die Ziele und Anforderungen im Rahmen des Projektes interpretiert werden.
In der Entwurfsphase werden für die Umsetzung der Anforderungen Lösungswege erarbeitet, verglichen und bewertet. Außerdem werden in diesem Abschnitt die einzelnen Komponenten des Frameworks spezifiziert sowie die Softwarearchitektur erstellt.
Im Kapitel Implementierung wird der Plan aus der Entwurfsphase umgesetzt. Auf besonders wichtige Implementierungsdetails wird hingewiesen und diese erläutert. 
Die prototypische Voice-over-IP Konsolenanwendung wurde parallel neben den Framework entwickelt und dient auschließlich zum Testen der Funktionalität. Da diese Anwendung auch ein Teil des Projektes ist, wird diese gesondert, in einem eigenen Kapitel, erörtert. Es wird erklärt wie die Funktionalitäten innerhalb der Anwendung getestet und umgesetzt worden sind.
Im letzten Kapitel wird ein Fazit gezogen. Es wird überprüft, ob alle Ziele und Anforderungen umgesetzt wurden. Außerdem gilt es zu klären, in welchen Umfang die Ziele erreicht wurden und wo es noch Optimierungsbedarf gibt. Im abschließenden Ausblick werden sinnvolle Erweiterungs- und Verbesserungsmöglichkeiten für das Framework vorgeschlagen.
	

\chapter{Anforderungsanalyse}
In diesem Kapitel sollen aus der Projektbeschreibung und die daraus resultierenden konkreten Anforderungen für das Framework erstellt werden. 

\section{Funktionale Anforderungen}
	In diesem Kapitel wird aus der Projektbeschreibung, die funktionalen Anforderungen ermittelt und definiert. Die funktionale Anforderungen werden in zusammenhängenden Gruppen eingeteilt.

	\begin{itemize} 
		\item Schnittstelle zur Audiohardware

		Es muss eine entsprechende Schnittstelle zur Audiohardware, Mikrofone und Lautsprecher, erstellt werden. Diese wird benötigt, um die Audiodaten zu übertragen und abzuspielen. Da ein Computer über mehr als ein Lautsprecher und Mikrofon verfügen kann, muss hier eine entsprechende Auswahlmöglichkeit existieren. Desweiteren müssen Konfigurationsmöglichkeiten vorhanden sein um z.B. bestimmte Abtastrate auszuwählen.
			
		\item Audiokompression

		Wie bereits erwähnt ist eine Audiokommunikation ohne entsprechende Datenkompression praktisch nicht realisierbar. Es muss eine allgemeine Schnittstelle für Dekodierer erstellt werden, die durch beliebige Audiocodecs erweitern lässt. Außerdem muss mindestens ein Codec integriert werden, um die Praxistauglichkeit des Frameworks zu gewährleisten.
			
		\item Netzwerkschicht

		Für den Verbindungsaufbau und -abbau sowie der Datenübertragung muss ebenfalls eine Schicht erstellt werden. Dabei müssen auch Verbindungsdaten verwaltet und verarbeitet werden. Die Übertragung basiert auf UDP.
				
		\item RTP-Protokoll

		Das Ent- und Verpacken von Nutzdaten in das RTP-Protokoll muss ebenfalls implementiert werden.
				
		\item Jitter-Buffer (RTPBuffer)

		Der Buffer hat die Aufgabe die Paketreihenfolge von den empfangenen Paketen wiederherzustellen, da diese nicht gewährleistet werden kann mit dem UDP-Protokoll. Dadurch erhöht sich die Audioqualität. Der Buffer soll einen bestimmten Füllstand erreichen, bevor die Audiowiedergabe beginnt.
				
		\item Prototypische Voice-over-IP Konsolenanwendungen

		Diese Anwendung soll alle Funktionen des Frameworks testen. Dadurch kann sichergestellt werden, dass das Framework die gewünschte Funktionalität tatsächlich umsetzt und anbietet. Die Hauptaufgabe der Testanwendung ist es, die Kommunikation zwischen zwei Anwendern zu ermöglichen. Die Anwendung soll sich mit Parameter konfigurieren lassen. 
				
		\item Statistik

		Um die Funktionen objektiv bewerten zu können, soll die Möglichkeit bestehen, statistische Werte zu erfassen und zu speichern.
		
	\end{itemize}
	
\section{Nicht-Funktionale Anforderungen}
Hier werden aus der Projektbeschreibung die nicht-funktionalen Anforderungen ermittelt und definiert.

\begin{itemize} 
\item Plattformunabhängigkeit

Ist einer der zentralen Ziele des Frameworks. Es soll unter Windows, Linux und OS X funktionieren. Daraus folgt, dass eine Kommunikation zwischen einen Windows-Client und Linux-Client, oder einer sonstigen beliebigen Konstellation funktioniert, da die Funktionsweise immer identisch ist. Diesbezüglich wurde der Einsatz von CMake, ein Buildtool für C++, festgelegt.
		
\item Performanz

Latenzen sollten für eine flüssige und störungsfreie Kommunikation so gering wie möglich gehalten werden. Daraus resultiert auch die Wahl der Programmiersprache C++. In der weiteren Entwicklungsphase gilt es diesen Aspekt mit erhöhter Priorität betrachten.
		
\item Prinzipien des objektorientierten Designs

Das Framework soll nach den Prinzipien guten objektorientierten Designs entwickelt werden. Darunter versteht man Konventionen die Beschreiben wie gute Software aussehen sollte. Das Framework soll modular aufgebaut sein, so dass die Komponenten auch einzeln nutzbar sind. Außerdem soll es möglich sein die Klassen zu erweitern und auszutauschen. Konfigurationsmöglichkeiten sollen möglichst zur Laufzeit möglich sein. 
		
\item Dokumentation

Das Erstellen einer Dokumenation ist ebenfalls eine Teilaufgabe des Projektes.
		
\end{itemize}

\newpage
\section{Übersicht aller Anforderungen}
Es wird eine Übersicht aller Anforderungen, mit kurzen Beschreibungen erstellt, um in den folgenden Kapiteln darauf referenzieren zu können.

\begin{compactenum}[a)]
	\item Schnittstelle zur Audiohardware
		\begin{compactenum}[1.]
			\item Erstellen der Schnittstelle
			\item Konfigurationsmöglichkeiten der Audiohardware
		\end{enumerate}
	\item Audiokompression
		\begin{compactenum}[1.]
			\item Erstellen der Schnittstelle
			\item Einbinden eines Audiocodecs
		\end{compactenum}
	\item Netzwerkschicht
		\begin{compactenum}[1.]
			\item Erstellen einer Netzwerkschicht
			\item Konfigurationsmöglichkeiten für Netzwerkverbindungen
		\end{compactenum}
	\item RTP-Protokoll
		\begin{compactenum}[1.]
			\item Implementierung des RTP-Protokolls
		\end{compactenum}
	\item Jitter-Buffer
		\begin{compactenum}[1.]
			\item Implementieren eines Jitter-Buffers
			\item Audiowiedergabe erst starten, wenn bestimmter Füllstand erreicht ist
		\end{compactenum}
	\item Prototypische Voice-over-IP-Konsolenanwendung
		\begin{compactenum}[1.]
			\item Basiert auf das Framework und ermöglicht die Audiokommunikation zwischen zwei Anwendern
			\item Nimmt Parameter beim Programmstart entgegen und verarbeitet sie
		\end{compactenum}
	\item Statistik
		\begin{compactenum}[1.]
			\item Erfassen von statistischen Werten muss möglich sein
			\item Formatierte Ausgabe der Ergebnisse
		\end{compactenum}


	\item Plattformunabhängigkeit mit CMake
		\begin{compactenum}[1.]
			\item Windows, Linux, OS X
		\end{enumerate}
		\item Performanz
		\begin{compactenum}[1.]
			\item Performanz hat erhöhte Priorität
		\end{enumerate}
		\item Prinzpien des objektorientes Designs
		\begin{compactenum}[1.]
			\item Erweiterbarkeit, Wiederverwendbarkeit, Austauschbarkeit der Komponenten
		\end{enumerate}
		\item Dokumentation
		\begin{compactenum}[1.]
			\item Erstellen einer Dokumentation
		\end{enumerate}
\end{compactenum}


\chapter{Entwurf}
\section{Projektverwaltung und Werkzeuge}
\section{Build Prozess}
\section{Externe Softwarekomponenten}
\section{Softwarearchitektur}
\subsubsection{RTAudio}
\subsubsection{Opus}
\subsubsection{CPPTest}
\subsection{Verarbeitungskette}
\subsection{Instanziierung}
\subsection{Statistiken}
\subsection{Einstellungsmöglichkeiten und Konfigurationen}
\subsection{RTP-Protokoll}
\subsubsection{RTCP-Protokoll}
\subsection{RTP-Buffer}

\chapter{Implementierung}

\chapter{Prototypische Voice-over-IP Konsolenanwendungen}
\subsection{Ziel der Anwendung}
\subsection{Steuerung}
\subsection{Anwendungen ordnungsgemäß beenden}

\chapter{Fazit und Ausblick}
\section{Projektergebnis}
TODO: Evtl auch Vergleich mit Zielen/Aufgabenstellung
\section{Anwendungsmöglichkeiten}
\section{Erweiterungsmöglichkeiten}
TODO: Derzeit in Planung/Entwicklung: SIP für Konfiguration, Codecs A-law, $\mu$-law und Amplifier als Prozessoren

TODO: Zusätzlich möglich: sämtliche Codecs als Prozessoren, Filter, Vrestärker, Konverter als Prozessoren, weitere Audio-handler


\bibliographystyle{apalike} % Literaturverzeichnis
\begin{btSect}{./sources/literatur} % mit bibtopic Quellen trennen
\section*{Literaturverzeichnis}
\btPrintCited
\end{btSect}
\begin{btSect}{./sources/online}
\section*{Online-Quellen}
\btPrintCited
\end{btSect}

\end{document}
