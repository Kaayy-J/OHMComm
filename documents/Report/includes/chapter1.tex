\chapter{Einleitung}
\section{Was ist OHMComm?}
OHMComm ist ein IT-Projekt der Informatik Fakultät an der technischen Hochschule in Nürnberg. Das Projekt wird im Rahmen des Bachelorstudienganges Informatik umgesetzt und erstreckt sich über zwei Semester. Herr Prof. Dr. M. Tessmann ist Projektinitiator und Projektleiter. Das Ziel des Projektes ist die Erstellung eines plattformunabhängiges Audiokommunikationsframework. Das Framework soll sämtliche Funktionalität, die für eine erfolgreiche Kommunikation zwischen zwei Teilnehmern benötigt wird, zur Verfügung stellen. Das Framework als auch das Projekt tragen den Namen OHMComm. Der Anwendungsfall für die Hochschule ist der Einsatz der Software in Lehrveranstaltungen, jedoch sind auch andere Anwendungsszenarien denkbar, wie z.B. die Integration in externe Softwareanwendungen.

\section{Projektbeschreibung}
Die Projektbeschreibung von Herr Prof. Dr. M. Tessmann beschreibt die Ziele, die im Rahmen des IT-Projektes erreicht werden sollten. Der obligatorische Funktionsumfang des Audiokommunikationsframework lässt sich in vier größere Bereiche zerlegen. Zum einem wird für die Audioaufnahme und Wiedergabe eine Schnittstelle zur Audiohardware benötigt. Der zweite Block beinhaltet die Kodierung und Dekodierung von Audiodaten. Diese Funktionalität darf man keineswegs als optional betrachten, da in der Praxis eine unkomprimierte Audioübertragung nicht machbar ist. Die meisten Leitungen verfügen hierfür nicht über die notwendige Bandbreite. Das Versenden und Empfangen von Audiodaten mit dem User Datagram Protocol (UDP) und Real-Time-Transport-Protokol (RTP) ist ein weiterer großer Implementierungsblock. RTP stellt ein standardisiertes Protokoll für die Audio- und Videoübertragung dar, welches 1996 von der Audio-Video Transport Working Group of the Internet Engineering Task Force (IETF) entwickelt wurde. Der letzte Block umfasst das Erstellen eines Jitterbuffers. Durch das Versenden von UDP-Paketen kann die Paketreihenfolge beim Empfänger nicht gewährleistet. Die Aufgabe des Buffers ist es, durch eine minimale Verzögerung vor dem Abspielen der Audiodaten, die Paketreihenfolge wiederherzustellen. Dadurch lässt sich die Audioqualität wesentlich erhöhen. Sämtliche Funktionen des Frameworks sollen in einer Beispielanwendung getestet werden, welche ebenfalls ein Teil des Projektes ist. Ferner soll es möglich statistische Werte des Frameworks zu sammeln, um es entsprechend bewerten zu können. Latenzen spielen in der Audioverarbeitung eine wesentliche Rolle, deshalb hat der Punkt Performanz eine besonderen Stellenwert. Auch das Ziel der Plattformunabhängigkeit ist ein wesentliches Ziel des Frameworks, das umgesetzt werden soll. Außerdem soll das Framework den Prinzipien des objektorientierten Designs entsprechen.
	
\section{Aufbau des Berichts}
Der Aufbau des Berichts entspricht den klassischen Phasen der Softwareentwicklung und ist unterteilt in den Kapiteln Anforderungsanalyse, Entwurf, Implementierung, Prototypische Voice-over-IP Konsolenanwendung und dem Fazit und Ausblick. 
In der Anforderungsanalyse werden aus den Projektzielen die konkreten Anforderungen ermittelt und diese in funktionale und nicht-funktionale Anforderungen klassifiziert. Außerdem wird hier definiert, wie die Ziele und Anforderungen im Rahmen des Projektes interpretiert werden.
In der Entwurfsphase werden für die Umsetzung der Anforderungen Lösungswege erarbeitet, verglichen und bewertet. Außerdem werden in diesem Abschnitt die einzelnen Komponenten des Frameworks spezifiziert sowie die Softwarearchitektur erstellt.
Im Kapitel Implementierung wird der Plan aus der Entwurfsphase umgesetzt. Auf besonders wichtige Implementierungsdetails wird hingewiesen und diese erläutert. 
Die prototypische Voice-over-IP Konsolenanwendung wurde parallel neben den Framework entwickelt und dient auschließlich zum Testen der Funktionalität. Da diese Anwendung auch ein Teil des Projektes ist, wird diese gesondert, in einem eigenen Kapitel, erörtert. Es wird erklärt wie die Funktionalitäten innerhalb der Anwendung getestet und umgesetzt worden sind.
Im letzten Kapitel wird ein Fazit gezogen. Es wird überprüft, ob alle Ziele und Anforderungen umgesetzt wurden. Außerdem gilt es zu klären, in welchen Umfang die Ziele erreicht wurden und wo es noch Optimierungsbedarf gibt. Im abschließenden Ausblick werden sinnvolle Erweiterungs- und Verbesserungsmöglichkeiten für das Framework vorgeschlagen.
	
